\section{Appendix G: Thermal Conductance and Anderson Localization}
\label{appendix:thermal}

The resilience of the DAT-E6 lattice against thermal dissipation is modeled via the localization of vibrational modes. Unlike periodic lattices where heat follows the Fourier law, the DAT-E6 structure exhibits a "fractal energy trapping" mechanism.

\subsection{Anderson Localization Parameters}
The localization of phonons is characterized by the Inverse Participation Ratio (IPR), defined as:
\[
\text{IPR} = \frac{\sum_{i} |\phi_i|^4}{(\sum_{i} |\phi_i|^2)^2}
\]
where $\phi_i$ represents the amplitude of the vibrational mode at site $. High IPR values in the DAT-E6 benchmarks indicate spatial confinement of energy.

\subsection{Resonance Jump and Phason Strain}
The stability singularity observed at $\delta_0 \approx 0.309$ corresponds to the structural resonance between the icosahedral projection and the fluid high-frequency modes. The phason strain energy $ is derived as:
\[
E_p = \frac{1}{2} C_{ij} \partial_i w_j \partial_j w_i
\]
where $ represents the phason displacement field. At $\delta_0$, the coupling between the phonon and phason fields minimizes heat leakage to the observed -zsh.004\%$.

\subsection{Simulation Verification}
Numerical validation was performed at {source} = 1000^\circ\text{C}$. The thermal conductivity $\kappa$ was found to be:
\[
\kappa_{DAT-E6} \approx 0.85 \, \text{W/mK}
\]
compared to $\kappa_{Al} \approx 235 \, \text{W/mK}$, confirming an orders-of-magnitude reduction in thermal transport due to the quasi-periodic potential.
